List of resources, code, and tutorials we used along the way

\begin{DoxyAuthor}{Author}
Grant Jurgensen, Stephen Longofono, Stephen Wiss
\end{DoxyAuthor}
\hypertarget{Citations_Citations}{}\section{Citations}\label{Citations_Citations}
\hypertarget{Citations_one}{}\subsection{Leap Year Algorithm}\label{Citations_one}
An algorithm to determine if it is a leap year, adapted from the logic shown in this wikipedia page\+: Accessed September, 2016, \href{https://en.wikipedia.org/wiki/Leap_year#Algorithm}{\tt https\+://en.\+wikipedia.\+org/wiki/\+Leap\+\_\+year\#\+Algorithm}\hypertarget{Citations_two}{}\subsection{Day of the Week Algorithm}\label{Citations_two}
An algorithm to determine the day of the week of the first day of the year, adapted from the algortihm on the \char`\"{}disparate variation\char`\"{} of Gauss\textquotesingle{}s algorithm, shown here\+: Accessed September, 2016, \href{https://en.wikipedia.org/wiki/Leap_year#Algorithm}{\tt https\+://en.\+wikipedia.\+org/wiki/\+Leap\+\_\+year\#\+Algorithm}\hypertarget{Citations_three}{}\subsection{Regular Expressions}\label{Citations_three}
Regular expressions were built for date parsing using a tutorial written by Jan Goyvaerts. Accessed September 2016, \href{https://www.regular-expressions.info/dates.html}{\tt https\+://www.\+regular-\/expressions.\+info/dates.\+html}\hypertarget{Citations_four}{}\subsection{Flask Extensions \& Best Practices}\label{Citations_four}
We consulted the official Flask documentation while writing our server, and adapted many of the examples to fit our needs. Since much of what we used was simply T\+HE way to implement any given feature, we did not cite them individually within the code.

Flask Documentation by Armin Ronacher, Accessed September 2016, \href{http://flask.pocoo.org/docs/0.11/}{\tt http\+://flask.\+pocoo.\+org/docs/0.\+11/}\hypertarget{Citations_five}{}\subsection{Flask Micro-\/\+Blog Tutorial}\label{Citations_five}
This extensive tutorial was used as a model for our authentication, and was also where we got the Bootstrap code for our client-\/side C\+SS.

The Flask Mega-\/\+Tutorial by Miguel Grinberg, Accessed September 2016, \href{http://blog.miguelgrinberg.com/post/the-flask-mega-tutorial-part-i-hello-world}{\tt http\+://blog.\+miguelgrinberg.\+com/post/the-\/flask-\/mega-\/tutorial-\/part-\/i-\/hello-\/world}\hypertarget{Citations_six}{}\subsection{Bootstrap C\+SS}\label{Citations_six}
As mentioned above, we sourced the original Bootstrap C\+SS from Miguel Grinberg\textquotesingle{}s Flask tutorial. The Bootstrap itself was written by unnamed employees of Twitter. Accessed September 2016

Their citation\+: Bootstrap Responsive v2.\+2.\+2

Copyright 2012 Twitter, Inc Licensed under the Apache License v2.\+0 \href{http://www.apache.org/licenses/LICENSE-2.0}{\tt http\+://www.\+apache.\+org/licenses/\+L\+I\+C\+E\+N\+S\+E-\/2.\+0}

Designed and built with all the love in the world  by  and .\hypertarget{Citations_seven}{}\subsection{Javascript String Sanitizing}\label{Citations_seven}
We used a function from a stackexchange post written by user \textquotesingle{}Arun P Johny\textquotesingle{} to trim out unwanted characters from the client-\/side form data. There is probably a better way to do so, but it eluded our Google-\/\+Fu.

Accessed September 2016, \href{https://stackoverflow.com/questions/16171320/remove-all-slashes-in-javascript#16171353}{\tt https\+://stackoverflow.\+com/questions/16171320/remove-\/all-\/slashes-\/in-\/javascript\#16171353}\hypertarget{Citations_eight}{}\subsection{Calendar Images}\label{Citations_eight}
The calendar images in the year view were taken from the website calendarpedia, under template 8

Accessed September 2016, \href{http://www.calendarpedia.com/2016-calendar-pdf-templates.html}{\tt http\+://www.\+calendarpedia.\+com/2016-\/calendar-\/pdf-\/templates.\+html} \href{http://www.calendarpedia.com/2017-calendar-pdf-templates.html}{\tt http\+://www.\+calendarpedia.\+com/2017-\/calendar-\/pdf-\/templates.\+html} 